\PassOptionsToPackage{svgnames}{xcolor}
\documentclass[twoside,palatino]{book}
%% Preamble:

% Set Watermark
%\usepackage{draftwatermark}
%\SetWatermarkText{\textmd{For PHPTR Review Only}}
%\SetWatermarkColor[gray]{0.96}
%\SetWatermarkScale{1.8}
%\SetWatermarkAngle{315}

% Load any macro files needed for this book:b
\usepackage {epsfig}
\usepackage {graphicx}
\usepackage{listings}
%\usepackage[bw,numbered,framed]{mcode}
\usepackage[numbered,framed]{mcode}
\usepackage {keyval}
\usepackage {lscape}	
\usepackage {makeidx}
\usepackage{longtable}
\usepackage {framed}
\usepackage{amssymb}
\usepackage{amsmath}
\usepackage{amsfonts}
\usepackage{latexsym}
\usepackage{lscape}
\usepackage{enumerate}
\usepackage{textcomp}
%%%%%%%%%%%%%%%%%%%%%%%%%%%%%%%%%%%%%%%%%%%%%%%%%%%%%%%%%%%%%%%%%%%%%%%%%%%%%%

%================Add image in items ============
\usepackage{enumitem}
\newcommand*{\ImgItem}[1]{%
  \raisebox{-.3\baselineskip}{%
    \includegraphics[
      height=\baselineskip, 
      width=\baselineskip,
      keepaspectratio]{#1}%
  }%
}
\newcommand{\probitem}{\refstepcounter{enumi}\item[\ImgItem{monitor.png}~\theenumi.]}
%==========================================
\usepackage{hyperref}
\hypersetup{pdfborder={0 0 0},
	colorlinks=true,
	linkcolor=DarkBlue,
	citecolor=DarkBlue,
	urlcolor=DarkBlue}
\urlstyle{same}
%\usepackage[number=none]{glossary}
\usepackage{wrapfig}
\usepackage{lscape}
\usepackage{rotating}
\usepackage{caption}
\usepackage{subcaption}
\captionsetup{format=plain, font=small, labelfont=bf}
\captionsetup[table]{labelfont=bf, labelsep=newline}
\usepackage{textcomp}
\usepackage{epstopdf}
\usepackage{stmaryrd}
\usepackage{multirow}
\usepackage{tabularx,ragged2e,booktabs,caption}
\usepackage{array}
\usepackage{afterpage}
\usepackage{geometry}
\usepackage{amsthm}
%%%%%%%%%%%%%%%%%%%%%%%%%%%%%%%%%%%%%%%%%%%%%%%%%%%%%%%%%%%%%%%%%%%%%%%%%%%%%%%%%%%%%%%%%%%
%This algorithm package is added by Rahul
%%%%%%%%%%%%%%%%%%%%%%%%%%%%%%%%%%%%%%%%%%%%%%%%%%%%%%%%%%%%%%%%%%%%%%%%%%%%%%%%%%%%%%%%%%%
\usepackage{tabularx,ragged2e,booktabs,caption}
\usepackage{float,lscape}
\usepackage{diagbox}
\usepackage{makecell}
\usepackage[chapter]{algorithm}
\usepackage{algpseudocode}
%\DeclareCaptionFormat{algorithm2e}{\vspace{0ex}{%
% \parbox[c][0.9em][c]{\textwidth}{#1#2#3}}\par\rule{\linewidth}{-2.5pt}}
%\captionsetup[algorithm]{format=plain, font=footnotesize, labelfont={footnotesize,bf}}
\usepackage{titletoc}
%\usepackage{tocloft}
\algnewcommand{\algorithmicgoto}{\textbf{go to}}%
\algnewcommand{\Goto}[1]{\algorithmicgoto~\ref{#1}}%
\newcommand{\suchthat}{\;\ifnum\currentgrouptype=16 \middle\fi|\;}
%\renewcommand{\thealgorithm}{}
%%%%%%%%%%%%%%%%%%%%%%%%%%%%%%%%%%%%%%%%%%%%%%%%%%%%%%%%%%%%%%%%%%%%%%%%%
%%------For creating environment for examples, theorems etc..--------%%%%
%%%%%%%%%%%%%%%%%%%%%%%%%%%%%%%%%%%%%%%%%%%%%%%%%%%%%%%%%%%%%%%%%%%%%%%%%
\usepackage{amsthm,color}
\usepackage{tikz}
\usepackage{tcolorbox}
\usepackage{lipsum}
\tcbuselibrary{skins,breakable}
\usetikzlibrary{shadings,shadows}
%
%\newenvironment{myexampleblock}[1]{%
%    \tcolorbox[beamer,%
%    noparskip,breakable,
%    colback=LightGreen,colframe=DarkGreen,%
%    colbacklower=LimeGreen!75!LightGreen,%
%    title=#1]}%
%    {\endtcolorbox}
%
%\newenvironment{myalertblock}[1]{%
%    \tcolorbox[beamer,%
%    noparskip,breakable,
%    colback=LightCoral,colframe=DarkRed,%
%    colbacklower=Tomato!75!LightCoral,%
%    title=#1]}%
%    {\endtcolorbox}

\newenvironment{myblock}[1]{%
    \tcolorbox[beamer,%
    noparskip,breakable,
    colback=LightBlue,colframe=LightBlue,%
    colbacklower=LightBlue,%
    title=#1]}%
    {\endtcolorbox}
%%%%%%%%%%%%%%%%%%%%%%%%%%%%%%%%%%%%%%%%%%%%%%%%%%
%Glossary PAckage
\usepackage[acronym]{glossaries}
%%%%%%%%%%%%%%%%%%%%%%%%%%%%%%%%%%%%%%%%%%%%%%%%%%%%%%%%%%%%%%%%%%%%%%%%%%%%
%%%%%%%           Setting up Example Environment
\usepackage[framemethod=TikZ]{mdframed}
\usepackage{xcolor}
\newcounter{example}
\renewcommand{\theexample}{\thesection.\arabic{example}}

%% define the style
\mdfdefinestyle{example}{%
    linecolor=blue,
    outerlinewidth=1pt,
    bottomline=true,
    leftline=false,rightline=false,
    skipabove=\baselineskip,
    skipbelow=\baselineskip,
    frametitle=\mbox{},
}
%% setup the environments
%%% with number
\newmdenv[%
    style=example,
    settings={\global\refstepcounter{example}},
    frametitlefont={\bfseries Example~\theexample\quad},
]{example}
%%% without number (starred version)
\newmdenv[%
    style=example,
    frametitlefont={\bfseries Example~\quad},
]{example*}
%%%%%%%%%%%%%%%%%%%%%%%%%%%%%%%%%%%%%%%%%%%%%%%%%%%%%%%%%%%%%%%%%%%%%%%%%%%%%%%
\renewcommand\qedsymbol{$\blacksquare$} %Dark square at the end of proof
%\usepackage{marginnote}
%%%%%%%%%%%%%%%%%%%%%%%%%%%%%%%%%%%%%%%%%%%%%%%%%%%%%%%%%%%%%%%%%%%%%%%%%%%%%%%%
%%%%%%%%---------For adding Appendices at the end of each chapter-------%%%%%%%%
%%%%%%%%%%%%%%%%%%%%%%%%%%%%%%%%%%%%%%%%%%%%%%%%%%%%%%%%%%%%%%%%%%%%%%%%%%%%%%%%%
\usepackage{etoolbox}
\pretocmd{\chapter}{\renewcommand\thesection{\thechapter.\arabic{section}}}{}{}
\newcommand\appsection{%
  \setcounter{section}{0}%
  \renewcommand\thesection{\thechapter.\Alph{section}}}
%--------------------------------------------------------------------------------
%\newcolumntype{Y}{>{\centering\arraybackslash}X}
%\newcolumntype{P}[1]{>{\centering\arraybackslash}p{#1}}
%%%%%%%%%%%%%%%%%%%%%%%%%%%%%%%%%%%%%%%%%%%%%%%%%%%%%%%%%%%%%%%%%%%%%%%%%%%%%%%%%%%%%%%%%%%
%%%%%%%%%%%%%%%%%%%%%%%%%%%%%%%%%%%%%%%%%%%%%%%%%%%%%%%%%%%%%%%%%%%%%%%%%%%%%%%%%%%%%%%%%%%
%\numberwithin{algorithm}{subsection}
%\makeindex
%\makeglossary
%\input table
%\input mydefs
%\input setps

% New commands and/or command redefinitions
%
% You can also place such commands in a macro file (e.g. mydefs.tex)
% and load them in the preamble with "input mydefs"
% Here are some examples:

\newcommand{\be}{\begin{equation}}           %---- begin numbered equation
\newcommand{\ee}{\end{equation}}             %---- end numbered equation
\newcommand{\dst}{\everymath{\displaystyle}} %---- use displaystle in eqs.
\newcommand{\Nuclide}[2]{${}^{#2}$#1}        %---- Nuclide macro
\renewcommand{\SS}[1]{${}^{#1}$}             %---- superscript in text
\newcommand{\STRUT}{\rule{0in}{3ex}}         %--- small strut
\newcommand{\dotprod}{{\scriptscriptstyle \stackrel{\bullet}{{}}}}

%% Document Parts
\newtheorem{eg}{Example}[chapter]
\newtheorem{theorem}{Theorem}[chapter]
\newtheorem{lemma}{Lemma}[chapter]
\newtheorem{proposition}{Proposition}[chapter]
\newtheorem{corollary}{Corollary}[chapter]
